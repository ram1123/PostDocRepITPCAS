\section{Systematic uncertainties} \label{section:Systematics}
This analysis takes into account systematic uncertainties from theoretical and experimental sources. The uncertainties on the signal and on the single Higgs backgrounds are modeled as scale or shape uncertainties. The scale uncertainties affect the yield of the processes and are treated as log-normal uncertainties, while the shape uncertainties are modeled as variations of the \mgg shape of the processes, i.e. the peak position or the width. The systematic uncertainty associated with the data-driven estimate of the continuum background is accounted for via the discrete profiling method. Given the small number of expected signal events compared to the backgrounds, the effect of the systematics uncertainties on the final results is expected to be small compared to the statistical ones. In case a systematic uncertainty affects processes in different channels, it is considered fully correlated across those channels. The following sources of systematic uncertainties are considered:

\begin{enumerate}

  \item \textbf{Theoretical uncertainties on the HH cross section}: The combined uncertainty on the QCD scale and on the top mass is taken into account, considering also its dependence from the value of $\kappa_{\lambda}$. For the SM signal this uncertainty amounts to $-23/+6\% $. The combined uncertainty on the PDF modeling and on the strong coupling constant is also considered with a value of $3\% $ \cite{Grazzini:2018bsd}. 

  \item \textbf{Theoretical uncertainties on the single Higgs cross sections}: Process-dependent uncertainties related to the QCD scale, the PDF modeling, and the strong coupling constant are taken into account for the ggH, ttH, VBF H, and VH processes.

  \item \textbf{Theoretical uncertainties on the Higgs boson branching ratios}: Such uncertainties are considered for both the single and the double Higgs processes. The considered uncertainties on the $H\rightarrow\gamma\gamma$, $H\rightarrow VV$, and $H\rightarrow bb$ branching ratios are approximately $2\% $, $1.5\% $, and $1.2\% $, respectively. 

  \item \textbf{Integrated luminosity}: A scale uncertainty is defined according to the luminosity measurements performed by the CMS experiment ~\cite{CMS-LUM-17-003,CMS-PAS-LUM-17-004,CMS-PAS-LUM-18-002}.

  \item \textbf{Trigger}: The trigger efficiency is measured from data with a tag and probe procedure using $Z \rightarrow ee$ events. The related uncertainty is uncorrelated between the three data taking years. An additional considered source of uncertainty is related to inefficiencies of the ECAL L1 trigger at $|\eta|>2$ experienced during 2016 and 2017. This is modeled as a purely rate-changing uncertainty.

%  \item \textbf{Pile-up reweigthing}: The distribution of the simultaneous number of interactions in the Monte Carlo simulation is corrected to match the one observed in the data through an event reweight procedure. The uncertainty on the number of pile-up events is estimated by varying the proton-proton inelastic cross section, 69.2 mb, within $\pm$ 4.6\%. This is modeled as a scale uncertainty uncorrelated between the three data taking years. 

  \item \textbf{Electron and muon reconstruction, identification and isolation efficiency}: These efficiencies are evaluated in data and simulation with tag-and-probe techniques using Drell-Yan events \cite{Khachatryan:2015hwa, Sirunyan:2018fpa}. Scale factors are derived and applied to the simulated events to improve the agreement of the efficiencies between simulation and data. The related uncertainty is purely rate-changing and uncorrelated between the three data taking years.

  \item \textbf{Photon identification}: The efficiency of the pre-selection on the photon identification MVA score is estimated in data and simulation with a tag-and-probe technique using Drell-Yan events. Scale factors are applied to correct for the difference between the data and the simulation. The related uncertainty is purely rate-changing  and uncorrelated between the three data taking years.        

%  \item \textbf{Electron energy scale and resolution}: Corrections for the energy scale observed in the data are applied to match the simulated one, and for the energy resolution observed in the simulation to match the one observed in the data. The systematic uncertainty on the electron energy scale and resolution is taken considering the POG provided resolution error and the uncertainty on the electron $p_T$. This uncertainty is uncorrelated between the three data taking years.

  \item \textbf{Photon shower shape}: Corrections for the imperfect modeling of the photon shower shape (and isolation) variables in simulation are applied to improve the agreement with the data. The impact of this uncertainty is estimated from the difference of the photon energy scale before and after the correction. This is modeled as a shape uncertainty which is correlated between the three years of the data taking.

  \item \textbf{Photon energy scale and resolution}: Corrections for the difference of the photon energy scale and resolution between data and simulation are derived using $Z \rightarrow ee$ events, with electron-photon differences accounted for as a systematic uncertainty. This uncertainty is uncorrelated between the three years of the data taking.

% Not sure if we apply this. Check: https://github.com/atishelmanch/flashgg/blob/HHWWgg_dev/Taggers/interface/HHWWggTagProducer.h 
%   \item \textbf{Di-photons vertex identification efficiency}: Scale factors are computed to account for the difference between data and simulation in number of events for which the correct vertex has been selected. 

  %https://twiki.cern.ch/twiki/bin/view/CMS/JECUncertaintySources#Recommendation_for_analysis
  \item \textbf{Jet energy scale and resolution}: Corrections for the differences in the measured jet energies between data and simulation are applied \cite{Khachatryan:2016kdb}. The impact of the corresponding uncertainties on the signal yield is evaluated by varying the corrected jets four-momentum within their respective per-jet uncertainties and propagating the effect to the final result. Several sources of uncertainty are considered, each with a specific level of correlation among the three years of data taking. 

  \item \textbf{B-tagging}: The difference in the b-tagging score distribution between data and simulation is corrected for with a reweight of the simulated events dependent on the jet $p_T$, $|\eta|$, and flavor \cite{Sirunyan:2017ezt}. The corresponding uncertainty is purely rate-changing and uncorrelated between the three years of data taking.

%   \item \textbf{b-tagging scale factors}: Applied to each jet as a function of jet $p_T$ and $|\eta|$ in order to account for the efficiency difference in b-tagging and misidentification 
%         between data and Monte Carlo simulation \cite{Sirunyan:2017ezt}. The systematic effect is estimated by shifting the scale factor up and down by $\sigma$. 
%         The \textbf{1a} method \cite{BtaginMethods} is used to predict the correct event yield in data by changing the weight of the selected Monte Carlo events. This uncertainty is uncorrelated between the three 
%         data taking years.



  
\end{enumerate}

The systematic uncertainties due to finite statistics of Monte Carlo samples for the HH signal are neglected.
