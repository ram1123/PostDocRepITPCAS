\chapter*{Abstract}
\addcontentsline{toc}{chapter}{Abstract}

% \noindent \lipsum[1-3].
% This report summarises the work done by Ram Krishna Sharma in the CMS group of the Institute of High Energy Physics,
% with Prof. Mingshui Chen, from September 2022 to December 2024.
% The main focus was on the double Higgs non-resonant production with fully Hadronic final state
% of WW$\gamma \gamma$, Higgs to 4 lepton differentical and fiducial cross-section measurement
% and vector boson scattering measurement with semi-leptonic WW final state.
% Other secondary project includes the studies with the resonant production of Higgs boson
% (performed in 2020 with 2017 data, again started with full run-2 data), CPPF, EGamma HLT studies.


The Higgs boson, a cornerstone of the Standard Model (SM), offers unique opportunities to probe the fundamental
structure of particle physics.
This thesis presents a comprehensive study of Higgs boson properties, searches for physics beyond the
Standard Model (BSM), and contributions to detector development and operation using data collected by
the CMS experiment at the CERN Large Hadron Collider (LHC).

A significant portion of this work focuses on the search for resonant Higgs boson pair production
(\(X \to HH \to WW\gamma\gamma\)), which provides insights into the Higgs self-coupling and potential
extensions of the SM.
Additionally, high-mass scalar searches (\(X \to ZZ\)) are performed to explore new resonances
predicted by BSM models.
Precision studies are conducted in the \(H \to 4\ell\) channel to measure differential and fiducial cross sections,
offering stringent tests of SM predictions.

To probe anomalous interactions, effective field theory (EFT) parameterizations and measurements of
anomalous triple gauge couplings (aTGC) are utilized.
This thesis also includes the measurement of the Higgs boson mass in the \(H \to \gamma\gamma\) channel,
providing one of the most precise determinations of this fundamental parameter.
Studies of associated Higgs production (\(VH\)) further test the Higgs boson's interactions with vector bosons.

Beyond physics analyses, contributions to the CMS detector include developments in electron and photon
reconstruction algorithms, studies of the High-Granularity Calorimeter (HGCAL) for the High-Luminosity LHC
upgrade, and participation in detector operations.

The results and conclusions presented in this thesis provide valuable insights into the properties of
the Higgs boson and offer constraints on new physics scenarios.
These contributions advance our understanding of the SM and pave the way for future discoveries at the LHC.

\vspace*{2em}

\noindent \textbf{Keywords:} {CMS experiment, VBS, HH, differential, fiducial measurement, CPPF}
