\chapter{The CMS Detector}
\label{ch:cms_detector}

A detailed description of the CMS detector, together with a definition of the coordinate system
used and the relevant kinematic variables, can be found in Ref.~\cite{Chatrchyan:2008zzk}.

The central feature of the CMS apparatus is a superconducting solenoid with a 6~\unit{m}
internal diameter, providing a magnetic field of 3.8~\unit{T}.
The solenoid volume contains the following key subsystems:
\begin{itemize}
    \item \textbf{Silicon Tracker:} Measures the trajectories of charged particles with high precision,
    providing excellent momentum resolution.
    \item \textbf{Electromagnetic Calorimeter (ECAL):} A lead tungstate crystal calorimeter for
    energy measurements of photons and electrons.
    \item \textbf{Hadron Calorimeter (HCAL):} A brass and scintillator calorimeter for measuring
    hadronic jets and energy deposits.
    \item \textbf{Forward Calorimeters:} Extend the pseudorapidity coverage of the detector.
    \item \textbf{Muon System:} Gaseous detectors embedded in the steel flux-return yoke outside
    the solenoid for muon identification and measurement.
\end{itemize}

\section*{Electromagnetic Calorimeter (ECAL)}

The ECAL consists of 75,848 lead tungstate crystals, covering the pseudorapidity regions
$\abs{\eta} < 1.48$ (barrel, EB) and $1.48 < \abs{\eta} < 3.0$ (endcaps, EE).
Preshower detectors with two planes of silicon sensors and a total of $3 X_0$ of lead are
located in front of each EE section.

The ECAL achieves excellent energy resolution:
\begin{itemize}
    \item In the barrel, the resolution for unconverted or late-converting photons in the tens of
    GeV range is about 1\%.
    \item For other photons, the resolution is about 1.3\% up to $\abs{\eta} = 1$, rising to
    about 2.5\% at $\abs{\eta} = 1.4$.
    \item In the endcaps, the resolution is about 2.5\% for unconverted or late-converting photons
    and between 3--4\% for the remaining photons~\cite{CMS:2015myp}.
\end{itemize}

The diphoton mass resolution, as measured in $\Higgs \to \gamma\gamma$ decays,
typically lies in the 1--2\% range, depending on the photon energy resolution in the ECAL and
the event topology~\cite{CMS:2020xrn}.

\section*{Trigger System}

Events of interest are selected using a two-tiered trigger system:
\begin{itemize}
    \item \textbf{Level-1 Trigger (L1):} Composed of custom hardware processors,
    it uses calorimeter and muon detector information to select events at a rate of up to 100~\unit{kHz}
    with a fixed latency of about 4~\mus~\cite{Sirunyan:2020zal}.
    \item \textbf{High-Level Trigger (HLT):} A farm of processors runs a version of the full event
    reconstruction software optimized for speed, reducing the event rate to around 1~\unit{kHz}
    for storage~\cite{Khachatryan:2016bia}.
\end{itemize}

\section*{Luminosity Measurement}

The integrated luminosities for the 2016, 2017, and 2018 data-taking years have uncertainties ranging
from 1.2\% to 2.5\%~\cite{CMS-LUM-17-003, CMS-PAS-LUM-17-004, CMS-PAS-LUM-18-002}.
The overall uncertainty for the full 2016--2018 data-taking period is 1.6\%.

\section*{Particle Reconstruction}

The CMS particle-flow algorithm~\cite{CMS-PRF-14-001} reconstructs and identifies individual particles
in an event by combining information from all sub-detectors:
\begin{itemize}
    \item \textbf{Photons:} Energy is obtained from the ECAL measurement.
    \item \textbf{Electrons:} Energy is determined using the combination of the electron momentum
    measured in the tracker, the ECAL cluster energy, and bremsstrahlung photon recovery.
    \item \textbf{Muons:} Energy is determined from the curvature of the track in the muon system.
    \item \textbf{Charged Hadrons:} Energy is determined from the tracker momentum and matching
    ECAL/HCAL energy deposits, corrected for calorimeter response.
    \item \textbf{Neutral Hadrons:} Energy is obtained from corrected ECAL and HCAL energy
    measurements.
\end{itemize}

This comprehensive detector design and reconstruction strategy enable CMS to achieve the precision
required for a broad range of physics analyses.
